\documentclass[12pt,spanish]{homework}
\usepackage[utf8]{inputenc}
\usepackage[T1]{fontenc}
%\usepackage{mathpazo}
\usepackage{graphicx}
\usepackage{booktabs}
\usepackage{listings}
\usepackage{enumerate} 
\usepackage{amsmath}
\usepackage{amssymb}
\usepackage[spanish,mexico]{babel}
\inputencoding{latin1}

\title{Variable exponencial}
\author{Emilio Balan,Amilcar Campos, Elian Carrasco, Citlali Guti�rrez, H�ctor Rodr�guez}
\date{8 de noviembre del 2020}
\institute{Universidad Aut�noma de Yucat�n \\ Facultad de Matem�ticas - UADY}
\class{Probabilidad 1 (Grupo B)}
\professor{Lic. Ernesto Guerrero Lara}

\begin{document}

\maketitle

\section*{Funci�n de densidad}


\section*{Caracter�sticas de la funci�n de densidad}



\section*{Calcular la funci�n de distribuci�n}


\section*{Calcular la esperanza matem�tica}
La esperanza para esta variable exponencial es la siguiente:
$$E(X)= \dfrac{a + b}{2}$$
\subsubsection*{Demostraci�n}
Sabemos que la funci�n de densidad se encuentra dado por:
$$
f_{X} (x)=\begin{cases}
{\lambda e^{-\lambda x}} & si \mbox{ $ x  > 0$}\\{0 }& \mbox{ en otro caso}\
\end{cases}
$$
De igual forma sabemos que la esperanza para una variable continua, se calcula mediante:
$$E(X)= \int \limits_{- \infty}^{\infty} xf_{X}(x)dx$$
As� que la esperanza nos quedar�a de la siguiente forma:
$$E(X)= \int \limits_{0}^{\infty} x \lambda e^{-\lambda x}dx$$
Esta integraci�n se resuelve por partes, donde:
\begin{itemize}
\item $u = x$
\item $v = -e^{-\lambda x}$
\item $du = 1$
\item $dv = \lambda e^{-\lambda x}$
\end{itemize}
As� que la integral nos quedar�a de la siguiente manera:
$$E(X)= x(-e^{-\lambda x})\Big|_0^{\infty} - \int \limits_{0}^{\infty} - e^{- \lambda x} dx = \dfrac{- e^{- \lambda x}}{\lambda} \Big|_0^\infty = \dfrac{1}{\lambda}$$


Sea X una variable aleatoria con distribuci�n exp$\lambda$, sabemos que la esperanza cuenta con las siguientes propiedades:\\
a) $E(X^2) = \dfrac{2}{\lambda ^2}$\\
b) $E(X^n) = \dfrac{n!}{x^n}$
\subsection*{Demostraci�n de la propiedad a)}
La funci�n de densidad se encuentra dado por:
$$
f_{X} (x)=\begin{cases}
{\lambda e^{-\lambda x}} & si \mbox{ $ x  > 0$}\\{0 }& \mbox{ en otro caso}\
\end{cases}
$$
As� como la esperanza para una variable continua, se calcula mediante:
$$E(X)= \int \limits_{- \infty}^{\infty} xf_{X}(x)dx$$
As� que la esperanza nos quedar�a de la siguiente forma:
$$E(X^2)= \int \limits_{0}^{\infty} x^2 \lambda e^{-\lambda x} dx$$

$$E(X)= \dfrac{x^3}{3(b-a)}\Big|_a^b = \dfrac{b^3 - a^3}{3(b - a)}= \dfrac{(b -a)(b^2 + ab + a^2)}{3(b - a)} = \dfrac{a^2 + ab + b^2}{3} $$
Esta integraci�n se resuelve por partes, donde:
\begin{itemize}
\item $u = x^2$
\item $v = \dfrac{-e^{-\lambda x}}{\lambda}$
\item $du = 2x$
\item $dv = \lambda e^{-\lambda x}$
\end{itemize}
As� que la integral nos quedar�a de la siguiente manera:
$$E(X)= -x^2 e^{-\lambda x}\Big|_0^{\infty} + \int \limits_{0}^{\infty} 2xe^{- \lambda x} dx = \dfrac{-2xe^{- \lambda x}}{\lambda ^2} \Big|_0^\infty = \dfrac{2}{\lambda ^2}$$
\subsection*{Demostraci�n de la propiedad b)}


\section*{Calcular la varianza}
La varianza se obtiene de la forma ya conocida; es decir, como la varianza de esos mismos valores. Expresada en t�rminos de momentos, la varianza ser�:
$$Var(X)=\dfrac{1}{\lambda ^2}$$
\subsection*{Demostraci�n}
Por informaci�n previa sabemos que:
$$Var(X) = E(X^2) - E^2(X)$$
Y por las propiedades de la esperanza, sabemos que esto es igual a:
$$= \dfrac{2}{\lambda ^2} - \dfrac{1}{\lambda ^2}$$
$$= \dfrac{1}{\lambda ^2}$$



\section*{Ejercicios propuestos}
\end{document}
