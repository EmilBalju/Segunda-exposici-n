\documentclass[12pt,spanish]{homework}
\usepackage[utf8]{inputenc}
\usepackage[T1]{fontenc}
%\usepackage{mathpazo}
\usepackage{graphicx}
\usepackage{booktabs}
\usepackage{listings}
\usepackage{enumerate} 
\usepackage{amsmath}
\usepackage{amssymb}
\usepackage[spanish,mexico]{babel}
\inputencoding{latin1}

\title{Variable normal}
\author{Emilio Balan,Amilcar Campos, Elian Carrasco, Citlali Guti�rrez, H�ctor Rodr�guez}
\date{8 de noviembre del 2020}
\institute{Universidad Aut�noma de Yucat�n \\ Facultad de Matem�ticas - UADY}
\class{Probabilidad 1 (Grupo B)}
\professor{Lic. Ernesto Guerrero Lara}

\begin{document}

\maketitle

\section*{Funci�n de densidad}


\section*{Caracter�sticas de la funci�n de densidad}



\section*{Calcular la funci�n de distribuci�n}


\section*{Calcular la esperanza matem�tica}
La esperanza para esta variable normal es la siguiente:
$$E(X)= \mu$$
\subsubsection*{Demostraci�n}
Sabemos que la funci�n de densidad se encuentra dado por:
$$
f_{X} (x)= \dfrac{1}{\sqrt{2\pi \sigma ^2}} e^{\dfrac{-(x - \mu)^2}{2\sigma ^2}}, x\in\mathbb{R}
$$
De igual forma sabemos que la esperanza para una variable continua, se calcula mediante:
$$E(X)= \int \limits_{- \infty}^{\infty} xf_{X}(x)dx$$
As� que la esperanza nos quedar�a de la siguiente forma:
$$E(X)= \int \limits_{- \infty}^{\infty}  \dfrac{x}{\sqrt{2\pi \sigma ^2}} e^{\dfrac{-(x - \mu)^2}{2\sigma ^2}} dx$$
Realizamos un cambio de variable tal que $z = \dfrac{x - \mu}{\sigma}$ y observamos que $dx = \sigma dz$, obtenemos:
$$E(X)= \int \limits_{- \infty}^{\infty}  \dfrac{(\mu + z\sigma)}{\sqrt{2\pi \sigma ^2}} e^{\dfrac{-z^2}{2}} \sigma dz$$
Y esto a la vez es igual a:
$$E(X) = \mu \int \limits_{- \infty}^{\infty} \dfrac{e^{\dfrac{-z^2}{2}}}{\sqrt{2 \pi}} dz + \sigma \int \limits_{- \infty}^{\infty} \dfrac{ze^{\dfrac{-z^2}{2}}}{\sqrt{2 \pi}} dz $$
La primera integral (sin el factor $\mu$) representa el �rea total bajo la la distribuci�n normal estandarizada donde X tiene una distribuci�n N(0,1) y, por lo tanto, es igual a 1, la segunda integral es igual a cero puesto que el integrando, llam�moslo g(z), tiene la propiedad de que $g(z)=-g(-z)$ y por lo tanto g es una funci�n impar.
$$\therefore E(X) = \mu$$


Sea X una variable aleatoria con distribuci�n N$(\mu,\sigma ^2)$, sabemos que la esperanza cuenta con la siguiente propiedad:\\
$E(X^2) = \mu ^2 + \sigma ^2$\\
\subsection*{Demostraci�n }
Haciendo nuevamente, $z = \dfrac{x - \mu}{\sigma}$, obtenemos:
$$E(X^2)= \int \limits_{- \infty}^{\infty}  \dfrac{(\mu + z\sigma)^2}{\sqrt{2\pi \sigma ^2}} e^{\dfrac{-z^2}{2}} \sigma dz$$
Y esto a la vez es igual a:
$$E(X^2) = \mu^2 \int \limits_{- \infty}^{\infty} \dfrac{e^{\dfrac{-z^2}{2}}}{\sqrt{2 \pi}} dz + \sigma \mu \int \limits_{- \infty}^{\infty} \dfrac{2ze^{\dfrac{-z^2}{2}}}{\sqrt{2 \pi}} dz +  \sigma^2 \int \limits_{- \infty}^{\infty} \dfrac{e^{\dfrac{-z^2}{2}}}{\sqrt{2 \pi}} dz$$
La primera integral (sin el factor $\mu^2$) representa el �rea total bajo la la distribuci�n normal estandarizada donde X tiene una distribuci�n N(0,1) y, por lo tanto, es igual a 1, la segunda integral es igual a cero puesto que el integrando, llam�moslo g(z), tiene la propiedad de que $g(z)=-g(-z)$ y por lo tanto g es una funci�n impar y por �ltimo el tercero sucede el mismo caso que el primero e igual es 1.
$$\therefore E(X^2) = \mu^2 + \sigma^2$$

\section*{Calcular la varianza}
La varianza se obtiene de la forma ya conocida; es decir, como la varianza de esos mismos valores. Expresada en t�rminos de momentos, la varianza ser�:
$$Var(X)=\sigma ^2$$
\subsection*{Demostraci�n}
Por informaci�n previa sabemos que:
$$Var(X) = E(X^2) - E^2(X)$$
Y por las propiedades de la esperanza, sabemos que esto es igual a:
$$=(\mu ^2 + \sigma ^2)- (\mu ^2)$$
$$= \sigma ^2$$



\section*{Ejercicios propuestos}
\end{document}
