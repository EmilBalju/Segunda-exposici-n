\documentclass[12pt,spanish]{homework}
\usepackage[utf8]{inputenc}
\usepackage[T1]{fontenc}
%\usepackage{mathpazo}
\usepackage{graphicx}
\usepackage{booktabs}
\usepackage{listings}
\usepackage{enumerate} 
\usepackage{amsmath}
\usepackage{amssymb}
\usepackage[spanish,mexico]{babel}
\inputencoding{latin1}

\title{Variables Uniformes Continua}
\author{Emilio Balan,Amilcar Campos, Elian Carrasco, Citlali Guti�rrez, H�ctor Rodr�guez}
\date{8 de noviembre del 2020}
\institute{Universidad Aut�noma de Yucat�n \\ Facultad de Matem�ticas - UADY}
\class{Probabilidad 1 (Grupo B)}
\professor{Lic. Ernesto Guerrero Lara}

\begin{document}

\maketitle

\section*{Funci�n de densidad}


\section*{Caracter�sticas de la funci�n de densidad}



\section*{Calcular la funci�n de distribuci�n}


\section*{Calcular la esperanza matem�tica}
La esperanza para esta variable uniforme continua es la siguiente:
$$E(X)= \dfrac{a + b}{2}$$
\subsubsection*{Demostraci�n}
Sabemos que la funci�n de densidad se encuentra dado por:
$$
f_{X} (x)=\begin{cases}
{\dfrac{1}{b - a}} & si \mbox{ $a < x  < b$}\\{0 }& \mbox{ en otro caso}\
\end{cases}
$$
De igual forma sabemos que la esperanza para una variable continua, se calcula mediante:
$$E(X)= \int \limits_{- \infty}^{\infty} xf_{X}(x)dx$$
As� que la esperanza nos quedar�a de la siguiente forma:
$$E(X)= \int \limits_{- \infty}^{\infty} \dfrac{x}{b - a} dx$$
$$E(X)= \dfrac{x^2}{2(b-a)}\Big|_a^b = \dfrac{b^2}{2(b - a)} - \dfrac{a^2}{2(b - a)} = \dfrac{b^2 - a^2}{2(b - a)} = \dfrac{b + a}{2} $$


Sea X una variable aleatoria con distribuci�n unif(a,b), sabemos que la esperanza cuenta con las siguientes propiedades:\\
a) $E(X^2) = \dfrac{a^2 + ab + b^2}{3}$\\
b) $E(X^n) = \dfrac{b^{n+1} - a^{n+1}}{(n+1)(b -a)}$
\subsection*{Demostraci�n de la propiedad a)}
La funci�n de densidad se encuentra dado por:
$$
f_{X} (x)=\begin{cases}
{\dfrac{1}{b - a}} & si \mbox{ $a < x  < b$}\\{0 }& \mbox{ en otro caso}\
\end{cases}
$$
As� como la esperanza para una variable continua, se calcula mediante:
$$E(X)= \int \limits_{- \infty}^{\infty} xf_{X}(x)dx$$
As� que la esperanza nos quedar�a de la siguiente forma:
$$E(X^2)= \int \limits_{- \infty}^{\infty} \dfrac{x^2}{b - a} dx$$
$$E(X)= \dfrac{x^3}{3(b-a)}\Big|_a^b = \dfrac{b^3 - a^3}{3(b - a)}= \dfrac{(b -a)(b^2 + ab + a^2)}{3(b - a)} = \dfrac{a^2 + ab + b^2}{3} $$


\subsection*{Demostraci�n de la propiedad b)}
La funci�n de densidad se encuentra dado por:
$$
f_{X} (x)=\begin{cases}
{\dfrac{1}{b - a}} & si \mbox{ $a < x  < b$}\\{0 }& \mbox{ en otro caso}\
\end{cases}
$$
As� como la esperanza para una variable continua, se calcula mediante:
$$E(X)= \int \limits_{- \infty}^{\infty} xf_{X}(x)dx$$
As� que la esperanza nos quedar�a de la siguiente forma:
$$E(X^n)= \int \limits_{- \infty}^{\infty} \dfrac{x^n}{b - a} dx$$
$$E(X)= \dfrac{x^{n+1}}{(n+1)(b-a)}\Big|_a^b = \dfrac{b^{n+1} + a ^{n+1}}{(n+1)(b + a)}$$


\section*{Calcular la varianza}
La varianza se obtiene de la forma ya conocida; es decir, como la varianza de esos mismos valores. Expresada en t�rminos de momentos, la varianza ser�:
$$Var(X)=\dfrac{(b-a)^2}{12}$$
\subsection*{Demostraci�n}
Por informaci�n previa sabemos que:
$$Var(X) = E(X^2) - E^2(X)$$
Y por las propiedades de la esperanza, sabemos que esto es igual a:
$$= \dfrac{a^2 + ab + b^2}{3} - \dfrac{(a + b)^2}{4}$$
$$=\dfrac{4a^2 + 4ab + 4b^2 - 3b^2 - 6ab - 3a^2}{12}$$
$$= \dfrac{a^2 - 2ab + b^2}{12}$$
$$= \dfrac{(b - a)^2}{12}$$



\section*{Ejercicios propuestos}



\end{document}
